\documentclass[12pt]{constitution}

\begin{document}

% TITLE
% =====

\title{The Constitution of the Southampton Robotics Outreach Society in the University of Southampton Students' Union}
\author{Southampton Robotics Outreach}
\maketitle

% CONTENT
% =======

\article{Adoption of the Constitution}
\label{article:adoption-constitution}

The unincorporated association and its property shall be managed by and administered in accordance with this Constitution.

% ---------------------------------------------------------

\article{Name}
\label{article:name}

The association's name is ``Southampton Robotics Outreach'', to be known as ``SRO'' and hereinafter `the Group'.

% ---------------------------------------------------------

\article{Objects}
\label{article:objects}

The objectives of the Group, `the Objects', are:

\begin{enumerate}
    \item To share our excitement about engineering with others
    \item To introduce younger people to the range of activities involved in an engineering project including:
    \begin{enumerate}
        \item Teamwork
        \item Planning
        \item Technical Design
        \item Technical Build
        \item Testing
    \end{enumerate}
    \item To use the medium of robotics to inspire the next generation to take their involvement with engineering to the next level
    \item To develop our own project management and technical engineering experience by leading others
\end{enumerate}

% ---------------------------------------------------------

\article{Membership}
\label{article:membership}

\begin{enumerate}
    \item Membership is open to natural persons, and is not transferable to anyone else.
    \item Membership is constituted in the following categories:
    \begin{enumerate}
        \item Full, open only to Full Members of the Students' Union;
        \item Associate, open to Associate and Temporary Members of the Students' Union, and to those students of the University who have exercised their right not to be members of the Students' Union.
    \end{enumerate}
    \item Only Full Members are entitled to be elected to the Committee, or to propose, discuss and vote at a General Meeting. These are the sole privileges afforded to the Full Members over any other category of Membership.
    \item The Group may charge a fee for admission to Membership, which may be set by a Meeting of the Committee.
    \item The Committee must keep a register of members (`the register') on the Student Groups Hub provided by the Students' Union at \url{www.susu.org}.
    \item The Committee may only refuse an application for Membership if, acting reasonably and properly, they consider it to be in the best interests of the Group to refuse the application.
    \item Membership may be terminated if:
    \begin{enumerate}
        \item The Member resigns by written notice to the Committee.
        \item Any sum due from the Member to the Group is not paid in full within six months of it falling due.
        \item A Member ceases to be qualified for their category of Membership.
        \item Membership is revoked by a resolution of the Members in a General Meeting or a Meeting of the Committee, in accordance with Article \ref{article:disciplinary-action}, `Disciplinary Action'.
    \end{enumerate}
\end{enumerate}

% ---------------------------------------------------------

\article{General Meetings}
\label{article:general-meetings}

\begin{enumerate}
    \item The General Meeting constitute the Group's highest decision ­making body, subject to the provisions of this Constitution.
    \item The Group must hold an Annual General Meeting (AGM) in each academic year and not more than fifteen months may elapse between successive AGMs.
    \item A General Meeting that is not an Annual General Meeting is called an Extraordinary General Meeting (EGM).
    \item The Committee may call an Extraordinary General Meeting at any time.
    \item The Committee must call an Extraordinary General Meeting if requested to do so in writing by at least five Full Members of the Group.
    \begin{enumerate}
        \item The Members' written request must state a complete agenda for the EGM.
        \item If the Committee do not hold an EGM within five days of their receipt of the Members' written request, the Members may proceed to hold an EGM in accordance with Article \ref{article:proceedings-general-meetings}, `Proceedings of General Meetings'.
    \end{enumerate}
\end{enumerate}

% ---------------------------------------------------------

\article{Proceedings of General Meetings}
\label{article:proceedings-general-meetings}

\begin{enumerate}
    \item Notice:
    \begin{enumerate}
        \item The minimum period of notice required to hold an Annual General Meeting is ten days.  The minimum period of notice required to hold an Extraordinary General Meeting is three days.
        \item The notice must specify the date, time and place of the General Meeting, and an agenda for the General Meeting.
        \item If the General Meeting is to be an AGM, the notice must say so, and must invite nominations in accordance with Article \ref{article:appointment-committee}, `Appointment of the Committee'.
        \item Notice must be given to all Members and to the Committee.
    \end{enumerate}

    \item Chairing:
    \begin{enumerate}
        \item General Meetings shall usually be chaired by the person who has been elected as President.
        \item If there is no such person or they are not present within fifteen minutes of the time appointed for the General Meeting, the Full Members present must elect one of their number to chair.
    \end{enumerate}

    \item Associate Members may speak at General Meetings with the permission of the meeting.

    \item Voting:
    \begin{enumerate}
        \item Every Full Member present at a General Meeting, with the exception of the Chair, shall be entitled to one vote upon every voting matter.  In the case of an equality of votes, the Chair shall have a casting vote.
        \item Decisions may only be made by at least a simple majority of votes at a quorate General Meeting.
        \item All voting shall be by a show of hands or secret ballot, at the discretion of the Chair.
        \item If a Full Member is not present at the meeting, they are entitled to submit their choice of candidate to the Chair ahead of the vote.
        \begin{enumerate}
            \item A Full Member must inform the Chair of their intention to not vote in person at least 24 hours before the vote.
            \item All choices for a vote must be announced at least 12 hours in advance of the vote.
        \end{enumerate}
    \end{enumerate}

    \item Minutes:
    \begin{enumerate}
        \item Minutes must be taken of all proceedings at a General Meeting, including the decisions made and where appropriate the reasons for the decisions.
        \item Minutes of a General Meeting shall be made available to all Members within seven days.
    \end{enumerate}

    \item Reports:
    \begin{enumerate}
        \item If the General Meeting is an AGM, the Chair may invite any of the Committee to offer a report of their activities whilst in office.
        \item The Treasurer must present the Group's accounts to the Members at the AGM.
    \end{enumerate}

    \item Resolutions:
    \begin{enumerate}
        \item Any Full Member may propose a resolution to be discussed and voted upon at a General Meeting.
    \end{enumerate}
\end{enumerate}

% ---------------------------------------------------------

\article{Officers and the Committee}
\label{article:officers-committee}

\begin{enumerate}
    \item The Group and its property shall be administered and managed by a Committee comprising the officers appointed in accordance with Article \ref{article:appointment-committee}, `Appointment of the Committee'.
    \item The Group shall have the following officers:
    \begin{enumerate}
        \item President. The President shall oversee the organisation and management of the Group and the Committee as a whole; ensure the officers' accountability to Members, the Committee, and the Students' Union; and represent the Group to all external interests.
        \item Secretary. The Secretary shall oversee the administration of the Group, take minutes at General Meetings and Meetings of the Committee, and maintain the register.
        \item Treasurer. The Treasurer shall oversee the financing of the Group, set the Group's budget, oversee the acquisition of sponsorship for the Group and maintain the accounts of the Group.
        \item Welfare Officer. The Welfare Officer shall oversee the personal welbeing of the Group’s Members, including organising small-scale social events for the Group’s Members and mediating between members where such action is necessary.
        \item SourceBots Director. The SourceBots Director shall oversee all activities executed under the ``SourceBots'' brand name.
        \item Summer School Lead. The Summer School Lead shall oversee the annual ECS Computing, Electronics and Robotics summer school, shall ensure a reasonable timeline for preparation is followed throughout the academic year, and shall ensure that tasks both for preparation and for the event itself are well planned and delegated. They shall correspond with the relevant ECS staff members, Smallpeice members, and Student Robotics members as needed. The Summer School Lead should also be available and on hand for the entire week of the summer school as the volunteer coordinator.
        \item Second Summer School Lead. The Second Summer School Lead is an optional committee position that isn’t needed to have a full committee but should be filled if possible. The Second Summer School Lead shall support the main Summer School Lead in their responsibilities.
        \item Marketing Director. The Marketing Director shall oversee the organisation of small outreach activities, the design, production, and distribution of promotional materials, promotion to prospective members, potential sponsors, and manage the society’s online presence.
        \item Student Robotics Liaison. The Student Robotics Liaison is reponsible for co-ordination of events and communication between Student Robotics, an external Charity no. 1163168 and the Group.
    \end{enumerate}

    \item No one may be appointed a member of the Committee if they have been disqualified from becoming a member of the Committee under the provisions of Article \ref{article:disciplinary-action}, `Disciplinary Action'.
    \item The number of the Committee must not be less than three, though is not subject to any maximum. There must always be:
    \begin{enumerate}
        \item a President;
        \item a Secretary;
        \item a Treasurer.
    \end{enumerate}

    \item An officer or ordinary member of the Committee shall cease to hold office if they:
    \begin{enumerate}
        \item cease to be a Full Member of the Group.
        \item resign by notice to the Group, or
        \item are removed from office by a resolution of the Members in General Meeting or a Meeting of the Committee, in accordance with Article \ref{article:disciplinary-action}, `Disciplinary Action'.
    \end{enumerate}
\end{enumerate}

% ---------------------------------------------------------

\article{Meetings of the Committee}
\label{article:meetings-committee}

\begin{enumerate}
    \item The Committee may regulate their proceedings as they think fit, subject to the provisions of this Article.
    \item Any member of the Committee may request the Secretary to call a Meeting of the Committee.
    \item The Secretary must call a Meeting of the Committee if requested to do so by a member of the Committee.
    \item Meetings of the Committee shall usually be chaired by the person who has been elected as President.
    \item The quorum for a Meeting of the Committee shall be three members of the Committee.
    \item No decision may be made by a Meeting of the Committee unless a quorum is present at the time the decision is made.
    \item Every member of the Committee, with the exception of the Chair, shall be entitled to one deliberative vote upon every voting matter. In the case of an equality of votes, the Chair shall have a casting vote.
    \item Decisions may only be made by at least a simple majority of votes at a quorate Meeting of the Committee.
    \item There shall be no absentee voting.
    \item Minutes must be taken of all proceedings at a Meeting of the Committee, including the decisions made.
\end{enumerate}

% ---------------------------------------------------------

\article{Appointment of the Committee}
\label{article:appointment-committee}

\begin{enumerate}
    \item The Full Members of the Group in General Meeting shall appoint the officers and ordinary members of the Committee by election.
    \begin{enumerate}
        \item Elections for the Committee shall be held at an Annual General Meeting.  By-elections for vacant offices shall be held at an Extraordinary General Meeting.
        \item An Alternative Vote system shall be used for all elections.
        \item In all elections Re-Open Nominations, `RON', shall be a candidate.  An election yielding a result of RON shall be re-run as a by-election.
    \end{enumerate}

    \item The count for elections shall be conducted publicly by the Chair of the General Meeting, who must do so accurately.  Should the Members in General Meeting be dissatisfied with the accuracy of the count, they may resolve as a Point of Order to have the election re-counted or, if they remain dissatisfied, re-run as a by-election.

    \item \begin{enumerate}
        \item A member of the Committee shall assume office with effect from the conclusion of the General Meeting of his or her appointment.
        \item A member of the Committee shall retire with effect from the conclusion of the AGM next after his or her appointment, but shall be eligible for re-election at that AGM.
    \end{enumerate}

    \item The Committee must update their committee information on the Student Groups Hub provided by the Students' Union at \url{www.susu.org} (or failing that inform the Students' Union's Student Groups Officer) within seven days.
    \item A retiring member of the Committee must transfer all relevant information and documentation to his or her newly-elected counterpart, or to the President, within fourteen days.
\end{enumerate}

% ---------------------------------------------------------

\article{Financial Management}
\label{article:financial-management}

\begin{enumerate}
    \item The Committee are jointly liable for the proper management of the Group's finances.
    \item The income and property of the Group must be applied solely towards the promotion of the objects.
    \item The members of the Committee are entitled to be reimbursed from the property of the Group or may pay out of such property only for reasonable expenses properly incurred by him or her when acting on behalf of the Group.
    \item The accounts of the Group, as maintained by the Treasurer, must be made available to the Students' Union upon request.
\end{enumerate}

% ---------------------------------------------------------

\article{Irregularities and Saving Provisions}
\label{article:irregularities-saving-provisions}

\begin{enumerate}
    \item Subject to sub-clause (b) of this Article, all acts done by a Meeting of the Committee shall be valid notwithstanding the participation in any vote of a member of the Committee:
    \begin{enumerate}
        \item who was disqualified from holding office;
        \item who had previously retired or who had been obliged by this Constitution to vacate office;
        \item who was not entitled to vote on the matter, whether by reason of a conflict of interests or otherwise.
    \end{enumerate}

    \item Sub-clause (a) of this Article does not permit a member of the Committee to keep any benefit that may be conferred upon him or her by a resolution of the Committee if the resolution would otherwise have been void, or if the Committee has not complied with Article \ref{article:conflicts-interest-loyalities}, `Conflicts of Interests and Conflicts of Loyalties'.
    \item The Members in General Meeting may only invalidate, as a Point of Order, a resolution or act of:
    \begin{enumerate}
        \item the Committee;
        \item the Members in a General Meeting;
    \end{enumerate}
    if it may be demonstrated that a procedural defect in the same has materially prejudiced a Member of the Group.
\end{enumerate}

% ---------------------------------------------------------

\article{Conflicts of Interest and Conflicts of Loyalties}
\label{article:conflicts-interest-loyalities}

\begin{enumerate}
    \item A Member of the Committee must:
    \begin{enumerate}
        \item declare the nature and extent of any interest, direct or indirect, which they have in any decisions of a Meeting of the Committee or in any transaction or arrangement entered into by the Group which has not been previously declared;
        \item absent himself or herself from any discussions of the Committee in which it is possible that a conflict will arise between his or her duty to act solely in the interests of the Group and any personal interest, including but not limited to any personal financial interest.
    \end{enumerate}

    \item Any member of the Committee absenting himself or herself from any discussions in accordance with this Article must not vote or be counted as part of the quorum in any decision of the Committee on the matter.
\end{enumerate}

% ---------------------------------------------------------

\article{Disciplinary Action}
\label{article:disciplinary-action}

\begin{enumerate}
    \item Disciplinary action may be taken against any Member of the Group as a consequence of conduct:
    \begin{enumerate}
        \item detrimental to the reputation of the Group or the Students' Union.
        \item opposed to the objects of the Group (see Article \ref{article:objects}) or the Students' Union.
        \item in contravention of any provision of this Constitution.
    \end{enumerate}
    \item Disciplinary action that may be taken against any Member may be, but is not limited to:
    \begin{enumerate}
        \item issue of a formal written warning.
        \item partial or total ban from certain Group activities.
        \item disqualification from becoming a member of the Committee.
        \item removal of a member of the Committee from office.
        \item temporary or permanent revocation of Membership.
        \item referral of the complaint to the Students' Union's Disciplinary Committee.
    \end{enumerate}

    \item It is the right of the subject of the complaint to choose to have the disciplinary matter heard by either the Members in General Meeting, or a Meeting of the Committee.  Either shall have the power to take disciplinary action, including but not limited to those measures set out in paragraphs (i) - (vi) inclusive in sub-clause (b) of this Article.

    \item Any disciplinary hearing must be conducted in an impartial, balanced, and fair manner, considering all representations on the matter.

    \item All disciplinary action must be subject to prior discussion with the Students' Union's Student Groups Officer.

    \item Members subject to disciplinary action have the right of appeal to the Students' Union's Student Groups Committee.

    \item A full report of all disciplinary action taken by the Group in the previous year must be presented at the AGM.
\end{enumerate}

% ---------------------------------------------------------

\article{Affiliation to External Organisations}
\label{article:affiliation-external-organisations}

\begin{enumerate}
    \item The Group may only become an affiliate of an external organisation if:
    \begin{enumerate}
        \item the aims of that organisation are in line with those of the Group;
        \item the Members derive a direct benefit from the affiliation;
        \item no Policy of the Students' Union is breached by the affiliation;
        \item a resolution to affiliate is passed by the Members in General Meeting.
    \end{enumerate}

    \item The Group's affiliation to an external organisation shall immediately lapse:
    \begin{enumerate}
        \item at the conclusion of each Annual General Meeting after affiliation, unless the Members in General Meeting resolve to re-affiliate at each AGM in accordance with sub-clause (a) of this Article.
        \item if a resolution to disaffiliate is passed by the Members in General Meeting.
    \end{enumerate}

    \item All external affiliations and disaffiliations must be reported to the Students' Union's Student Groups Committee within seven days.
    \item For the avoidance of doubt, the Students' Union is not an external organisation for the purposes of this Article.
\end{enumerate}

% ---------------------------------------------------------

\article{Amendment to the Constitution}
\label{article:amendment-constitution}

\begin{enumerate}
    \item The Group may amend any provision contained in this Constitution provided that:
    \begin{enumerate}
        \item amendments do not:
        \begin{enumerate}
            \item alter the objects in such a way that undermines or works against the previous objects of the Group;
            \item retrospectively invalidate any prior act of the Members in General Meeting or a Meeting of the Committee;
        \end{enumerate}

        \item a resolution to amend a provision of this Constitution is passed by at least a two-thirds majority of the Full Members present at a General Meeting;
        \item a copy of the resolution amending this Constitution is sent to the Students' Union within seven days of it being passed;
        \item the resolution is ratified by the Students' Union's Student Groups Committee.
    \end{enumerate}

    \item The interpretation of this Constitution shall be with the Committee, except that during a General Meeting or a Meeting of the Committee the Chair shall have this responsibility.  The Members in General Meeting may resolve to revise any interpretation made by the Committee or a Chair as a Point of Order.
    \item The provisions of this Constitution shall be subordinate to those of the Articles, Rules, By-Laws and Policies of the Students' Union.
    \item The Committee and the Students' Union shall retain a copy of this Constitution, which the Committee must make available to Members upon request.
\end{enumerate}

% ---------------------------------------------------------

\article{Dissolution}
\label{article:dissolution}

\begin{enumerate}
    \item If the Members resolve to dissolve the Group, the Committee will remain in office and be responsible for winding up the affairs of the Group in accordance with this Article.
    \item A resolution to dissolve the Group must be passed by at least a two-thirds majority of the Full Members present at a General Meeting;
    \item The Committee must collect in all the assets of the Group and must pay or make provision for all the liabilities of the Group.
    \item The Committee must apply any remaining property or money:
    \begin{enumerate}
        \item directly for the objects;
        \item by transfer to any Group or Societies for purposes the same as or similar to the Group;
        \item in such other manner as the Students' Union's Student Groups Committee may approve in writing in advance.
    \end{enumerate}

    \item The Members may pass a resolution before or at the same time as the resolution to dissolve the Group specifying the manner in which the Committee are to apply the remaining property or assets of the Group.  The Committee must comply with such a resolution if it is consistent with the provisions of this Article.
    \item In no circumstances shall the net assets of the Group be paid to or distributed among the Members of the Group.
    \item The Committee must ensure the register and all other data held by the Group are securely destroyed upon the dissolution of the Group.
    \item The Committee must notify the Students' Union within seven days that the Group has been dissolved.  If the Committee are obliged to send the Group's accounts to the Students' Union for the accounting period which ended before its dissolution, they must send the Students' Union the Group's final accounts.
\end{enumerate}

% ---------------------------------------------------------

\article{Interpretation}
\label{article:interpretation}

In this Constitution:
\begin{enumerate}
    \item `The University' means `the University of Southampton'.
    \begin{enumerate}
        \item `University term' and `academic year' have the definitions set out in the University Calendar and Almanac.
    \end{enumerate}

    \item `Financial benefit' means a benefit, direct or indirect, which is either money of has monetary value.
    \item `The Students' Union' mean `The University of Southampton Students' Union'.
    \begin{enumerate}
        \item `Articles', or `Articles of the Students' Union' mean the Students' Union's Articles of Association. `Rules' and `Policies' have the definitions set out in the Articles. `By-Laws' has the definition set out in the Rules.
    \end{enumerate}
\end{enumerate}

% ---------------------------------------------------------

\article{Declaration}
\label{article:declaration}

The Members of the Group in a General Meeting adopted this Constitution:

Date:

President:

Secretary:

The Students' Union approved this Constitution:

Date:

Student Groups Officer:


\end{document}
